\chapter*{Resumen extendido}
\addcontentsline{toc}{chapter}{Resumen extendido} 

En este texto exponemos un estudio sobre la resolución del problema de ensamblado de genoma llamado \emph{de novo} usando el enfriamiento cuántico, un fenómeno físico utilizado por los ordenadores cuánticos de D-Wave.

En el primer capítulo estudiamos un modelo de mecánica cuántica desde un punto de vista matemático y teniendo en cuenta la posterior aplicación en computación cuántica. Explicamos cada postulado de la mecánica cuántica siguiendo un proceso iterativo. Para ello comenzamos estudiando la base matemática necesaria para entender el postulado. Tras esto, explicamos el postulado en detalle y damos una perspectiva de las implicaciones del mismo en la computación cuántica.

En particular, comenzamos explicando la notación de Dirac y los espacios de Hilbert para enunciar el primer postulado sobre el espacio de estados. Esto nos permite definir el qubit como entidad matemática independiente de su implementación física. Continuamos definiendo el producto vectorial, así como los operadores unitarios y Hermíticos para enunciar el postulado de la medida. Esto nos permite explicar la medición en los qubits.

A continuación definimos los valores y vectores propios de una matriz, lo que nos permitirá enunciar el tercer postulado sobre la evolución de un sistema cuántico. Explicamos la ecuación de Schrödinger y la implicación directa de esta evolución sobre la computación cuántica: las puertas cuánticas. Para el cuarto postulado definiremos el producto tensorial. Tras ello enunciamos dicho postulado sobre sistemas compuestos, así como la medida proyectiva, lo que nos permitirá entender sistemas con múltiples qubits, la medida en los mismos y el entrelazamiento cuántico.

Cerramos este primer capítulo enunciando y demostrando el teorema de no clonación. Este teorema, junto con el entrelazamiento cuántico, dictan las mayores disparidades entre la computación clásica y la cuántica.

En el segundo capítulo estudiamos cómo funcionan y cómo pueden utilizarse los ordenadores cuánticos de D-Wave. Para ello comenzamos entendiendo el efecto túnel, también conocido como \emph{quantum tunneling}, y la evolución adiabática: dos procesos físicos en los que se basan las arquitecturas de D-Wave. A continuación definimos los problemas cuadráticos de optimización binaria sin restricciones (QUBO), y presentamos una serie de ejemplos de traducción de problemas NP-duros clásicos a este modelo, como el problema del coloreado de grafos o el problema del viajante de comercio. Proporcionamos ejemplos numéricos de dichas transformaciones para todos los problemas estudiados. Más adelante veremos como el problema del ensamblaje del genoma puede transformarse en un problema del viajante de comercio. Es por esto que dicho problema cobra especial importancia, y dedicamos el debido tiempo a comprenderlo en profundidad. En particular, estudiamos un ejemplo que será posterior caso de estudios en los experimentos.

Introducimos finalmente las arquitecturas D-Wave, explicando la implementación física de sus qubits, así como el uso que realizan de la evolución adiabática y el efecto túnel. Tras esto explicamos cómo utilizar este tipo de arquitecturas para resolver problemas QUBO. Dedicamos una sección al estudio de las topologías de las QPUs de D-Wave. Tienen más relevancia para nuestros propósitos que el diseño de una CPU pues para resolver un problema utilizando estas arquitecturas hemos de embedir el grafo asociado al problema en la topología de la QPU.

En el tercer capítulo abordamos el problema del ensamblaje del genoma. Comenzamos exponiendo brevemente qué es el ensamble del genoma, sus dificultades y las aproximaciones clásicas conocidas hasta la fecha. Utilizaremos el método de \emph{overlap-layout consensus} (OCL) para transformar nuestro problema a un viajante de comercio, que es posteriormente codificado como un modelo QUBO. Finalmente, este proceso es puesto a prueba utilizando tanto enfriamiento simulado como los ordenadores cuánticos de D-Wave en una serie de experimentos expuestos al final de este trabajo. 

Comenzamos la fase de experimentación implementando y testeando las transformaciones necesarias para convertir el problema del genoma a un modelo QUBO, y lo resolvemos posteriormente utilizando tanto fuerza bruta como enfriamiento simulado. Para ello nos basaremos en el ejemplo de transformación a QUBO del problema del viajante de comercio comentado con anterioridad. Continuamos utilizando un ordenador cuántico D-Wave 2000Q para resolver el mismo problema y comparamos resultados. Tras estudiar este ejemplo en miniatura, implementamos una generación de tests aleatorios y ponemos a prueba la última máquina creada por D-Wave Systems, el sistema Advantage, en problemas de tamaño creciente. Los últimos experimentos están dedicados a la mejora de los resultados del sistema Advantage modificando los parámetros y la configuración en la ejecución del mismo.

Cerramos el texto presentando una serie de conclusiones y líneas de trabajo futuras.

\textbf{Palabras clave}: modelo de mecánica cuántica, enfriamiento cuántico, algoritmos cuánticos de optimización, problemas QUBO, ensamblaje de genoma