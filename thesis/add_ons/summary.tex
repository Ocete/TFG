\chapter*{Summary}
\addcontentsline{toc}{chapter}{Summary} 

In this text, we present a study on how to tackle the de novo genome assembly problem using quantum annealing.

In the first chapter, we develop a quantum mechanics model from a mathematic and quantum computing perspective. The quantum mechanics postulates are explained one by one through an iterative process: we begin by introducing the mathematical concepts required to understand the postulate. After that, the postulate is introduced and explained. Finally, we explore this postulate's implications from a quantum computing perspective. To finalize the chapter we state and prove the no-cloning theorem, one of the most important theorems in quantum computing.

In the second chapter, the gained based on quantum mechanics allows us to study quantum annealing and adiabatic evolution. These are the physical phenomenons on which D-Wave quantum computers are based. Next, Quadratic Unconstrained Binary Optimization (QUBO) problems are explained in detail, providing thorough examples on how to transform well-known NP-hard problems into QUBO models. Lastly, the D-Wave architectures are explained in-depth: how qubits are implemented, how quantum annealing is used to solve problems with these computers, and how we may embed a QUBO model into these systems.

In the third chapter, the de novo genome assembly problem is tackled. The problem is defined and then transformed into a traveling salesman problem. Using the heavy machinery developed in the previous chapter we are able to transform the de novo genome assembly problem into a QUBO, which is then embedded into a quantum annealer. Finally, a series of experiments using D-Wave quantum systems are exposed, using them to solve the genome assembly problem.

\textbf{Keywords}: quantum mechanics model, quantum annealing, quantum optimization algorithms, quadratic unconstrained binary optimization problems, de novo sequencing