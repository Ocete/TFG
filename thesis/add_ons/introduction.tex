\chapter*{Introduction and objectives}
\addcontentsline{toc}{chapter}{Introduction and objectives} 

Nowadays, the term \emph{quantum computing} sounds familiar not only to experts but also to the general public. Quantum computing theory began in 1980 when the physicist Paul Benioff proposed a quantum mechanical model of a Turing machine \cite{Benioff1980}. With Peter Shor's development of his integer factoring algorithm in 1994 \cite{Shor1999}, along with the work of several other researchers such as Richard Feynman and Yuri Manin\cite{Feynman1982} \cite{Manin1980}, it was clear that quantum computers had the potential to compute things that a classical computer could not feasible do.

However, it was not until 1995 with Cirac and Zoller's experiments of cold trapped ions that physical implementations of quantum machines were believed by most of the scientific world \cite{Cirac1995}. In recent years, public and private funding to quantum computing research has risen by $400\%$ \cite{Gibney2019}, and in October 2019, Google AI in a partnership with NASA claimed to have performed "quantum computations infeasible on any classical computer" \cite{Pednault2019}. As quantum computing becomes a day-to-day reality, mathematicians, computer scientists, and physicians developt and explore this growing field.

In particular, quantum annealers are a type of quantum computer focussed on optimization. D-Wave Systems is a private commercial quantum company that provides a series of quantum annealers for commercial and educational use. These quantum annealers have proven to be empirically useful for several real-world applications such as control of automated guided vehicles, traffic flow optimization or optimized planning scheduling \cite{Neukart2017} \cite{Ohzeki2019} \cite{Desimone2019}.

On the other hand, knowing an organism genome is key in multiple fields, including microbiology, biotechnology, and personalized medicine \cite{Costessi2018} \cite{Ginsburg2009} \cite{Abdallah2015}. Its applications range from the production of improved crops, biofuels, and pharmaceutical products to more precise complex disease detection and analysis. \emph{Genome assembly} refers to the process of, given a set of short genome reads, reconstruct the whole genome. This proceeding is a key component of the mentioned applications and as an NP-hard problem supposes a huge bottleneck for those developments. 

For the reasons explained above and inspired by the works \cite{Sohn2018} and \cite{Sarkar2020}  we set out to first gain a deep understanding of quantum computing and quantum annealers and then tackle the de novo genome assembly problem using D-Wave's quantum annealers. The objectives of this work are:

\begin{enumerate}
	\item To define and understand the theory behind quantum computing: quantum mechanics models. This shall be approached from a mathematic perspective and maintaining perspective of our subsequent application on quantum computing.
	
	\item To deeply understand how quantum annealers work, and how to use them to solve problems. This includes understanding quadratic unconstrained binary optimization (QUBO) problems, and how to transform NP-hard problems into such models.
	
	\item To theoretically and empirically tackle the de novo genome assembly problem, performing a comparative study between simulated annealing and quantum annealing using D-Wave's annealers.
\end{enumerate}

