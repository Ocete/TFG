\ctparttext{
	\color{black}
	\begin{center}
		Esta es una descripción de la parte de matemáticas.
		Nótese que debe escribirse antes del título
	\end{center}
}

\part{Parte de matemáticas}

\chapter{Sección primera}

Lorem ipsum dolor sit amet, consectetur adipiscing elit. Nunc finibus
augue a tellus volutpat, quis cursus mi egestas. Suspendisse cursus
eleifend lacus non consectetur. Proin nibh nisl, tincidunt eu
tincidunt et, auctor vitae mi. Mauris tincidunt finibus enim, sit amet
facilisis mi imperdiet in. Pellentesque pellentesque felis sed
condimentum congue. Nunc at mauris et velit tempor gravida faucibus
sed velit. Sed risus justo, feugiat non accumsan vitae, commodo a
mauris. Vestibulum ac tortor ligula. Donec pulvinar neque ac purus
varius dapibus. Donec fermentum bibendum ultrices. Quisque dictum,
purus quis semper lacinia, justo nibh cursus orci, vitae sagittis
mauris ex aliquet odio. Nullam vel lacus eu tellus dictum tempus in
vel sapien. Morbi pulvinar tincidunt tincidunt. Cras eros tortor,
commodo at tempus a, ultrices vel magna. Nulla eget mauris
arcu. Vivamus aliquet sem odio, non faucibus ante iaculis non.

Aenean aliquet metus nisi, eu congue nulla egestas nec. Etiam pretium
cursus orci a venenatis. Pellentesque placerat feugiat tortor tempus
laoreet. Sed at rutrum leo. Aenean magna nulla, egestas quis efficitur
a, consequat id ligula. Nulla facilisi. Duis orci urna, bibendum vel
mollis non, commodo vitae tellus. Integer luctus ex sed nibh
convallis, id malesuada metus facilisis.

\begin{theorem}
En las condiciones dadas anteriormente se tiene
\[
g(t) = \int^b_a K(t,s)f(s)\ \diff s.
\]
\end{theorem}

Vivamus sit amet sodales elit, in tempus nulla. Nullam euismod rhoncus
quam, ac fringilla massa fermentum vehicula. Maecenas vestibulum purus
non pellentesque congue. Sed nec massa purus. Mauris et quam
dignissim, mollis lacus id, tristique diam. Vestibulum molestie
vehicula tellus quis dignissim. Curabitur accumsan augue ut dictum
semper. Proin tristique quam nulla, vel placerat tortor feugiat
vel. Vestibulum quis ultricies enim, in vestibulum ante.

\chapter{Sección segunda}
Vivamus fringilla egestas nulla ac lobortis. Etiam viverra est risus,
in fermentum nibh euismod quis. Vivamus suscipit arcu sed quam dictum
suscipit. Maecenas pulvinar massa pulvinar fermentum
pellentesque. Morbi eleifend nec velit ut suscipit. Nam vitae
vestibulum dui, vel mollis dolor. Integer quis nibh sapien.

Nullam laoreet augue at erat consectetur fermentum. In interdum
aliquam condimentum. Etiam luctus viverra tellus, a consequat justo
venenatis id. Curabitur a scelerisque erat. Nulla et ante eget lacus
varius accumsan. Suspendisse odio nisl, convallis ac orci nec,
efficitur semper enim. Pellentesque eros ipsum, luctus et viverra
maximus, auctor at est. In hac habitasse platea dictumst. Etiam
ultricies suscipit diam, eget dignissim tellus. Proin justo orci,
volutpat placerat libero a, gravida interdum nisi.

\begin{corollary}
Se obtiene el siguiente resultado
\[\begin{tikzcd}
a\rar{f} \dar[swap]{g} & b \dar{h} \\
c\rar{k} & d
\end{tikzcd}\]
\end{corollary}
\begin{proof}
  Vivamus sit amet sodales elit, in tempus nulla. Nullam euismod
  rhoncus quam, ac fringilla massa fermentum vehicula. Maecenas
  vestibulum purus non pellentesque congue. Sed nec massa purus.
  Quod erat demostrandum.
\end{proof}
